\begin{comment}

\end{comment}

\section{Abstract}
% This thesis documents the design and implementation of a DSCS III Single Channel Transponder (SCT) beacon telemetry display. The system is a personal computer based design which interfaces to three SCT beacon receiver/demodulators. The software was designed to decode and display both the DSCS III A and DSCS III B satellite beacons. Recordings of the SCT beacon display can be made on paper and/or magnetic media when triggered by the user, a watchdog timer, or the SCT command accept telemetry bit. In addition, the system can be configured with an IRIG B Universal Time Coordinates (UTC) card which enables the software to determine the difference between the decoded SCT clock time and the local IRIG time source. Remoting the SCT configuration display is also possible using Hayes-compatible modems over a telephone link.

This report documents the design and implementation of an Open Mission Control Technologies (Open MCT) \gls{telemetry} storage and visualisation system for the NTNU \acrshort{hypso} SmallSat. Open MCT is a  The system is based on an \Gls{express} web server running on \Gls{node}, which requests and stores telemetry from a telemetry database operated by \Gls{nanoavionics}. The system was initially designed to decode Flight Computer and Electronic Power Supply module telemetry, but is expandable to accept other telemetry sources with minimal modifications to the Open MCT interface. It can store and subsequently display telemetry from these sources in near real-time in Open MCT. The stored telemetry data may also be exported or requested outside Open MCT by connecting to the \Gls{express}-based telemetry web server using \acrshort{http} for historical telemetry data, or with \Gls{ws} for subscribing to near real-time telemetry data updates.

The latest version of the documentation and source code for the system is available at \url{https://github.com/NTNU-SmallSat-Lab/mct-depot}.

\clearpage