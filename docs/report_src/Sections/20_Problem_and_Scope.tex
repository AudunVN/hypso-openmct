\begin{comment}
Assignment has no clear scope or specification, exploratory to some degree since much of it is unknown - set own requirements

Design Open MCT implementation; get telemetry from different sources, parse and adapt for display

Minimize overhead with adding new data sources - should require work in a small amount of new code; backend and frontend needs to be flexible

Prioritize loading and working with current known data sources (NA telemetry)
Requirements
 - Should be able to load current NA telemetry from their DB
 - Needs to be modular and maintainable
 - Should be able to provide x amount of data to x users in x time
 
What do we have to work with - how do we get telemetry, what form does it come in?
 - C struct stored in hex string

What restrictions do we have on how/when/where we can get the telemetry?
 - Restricted by IP, cannot get telemetry from anywhere easily - need to go through NA contact person to get new machines added

What does Open MCT need?
 - Object provider
 - Telemetry provider
 - Telemetry metadata provider
\end{comment}

\section{Problem and Scope} \label{scope}
The assignment for this project was fairly open and to some degree exploratory. It has therefore been necessary to collect some information about what was wanted from a telemetry processing and display system for HYPSO, such as which types of telemetry it should handle. This information has been collected and summarised in the sections below, and will be referenced later to guide the design and implementation process.

\subsection{Inputs}
The initial main source of data for the system and Open MCT will be stored historical telemetry data from a \Gls{postgrest} web server hosted by \Gls{nanoavionics}. Received telemetry values can be retrieved from this service as C-style packed \glspl{struct} stored in hexadecimal strings, along with some \gls{metadata}, inside a JSON object, as seen in Source Code \ref{code:eps_sample}. The telemetry returned from this server before HYPSO is launched comes from a ground-based test satellite \cite{na_postgrest}.

This server places fairly severe limitations on where it can be accessed from, with manually distributed API keys and IP address whitelisting being required before it can be used.

\begin{code}
\inputminted[linenos=true,breaklines=true,bgcolor=codebg]{javascript}{./Code/eps_telemetry_sample_trimmed.json}
\captionof{listing}{Sample EPS telemetry datum from NanoAvionics' PostgREST server (JSON)}
\label{code:eps_sample}
\end{code}

The data string (as seen in \inlinecode{epsa\_entry\_data} in Source Code \ref{code:eps_sample}) needs to be \glslink{unpacking}{unpacked} using a provided C \gls{struct} definition (see Source Code \ref{code:eps_struct}) before the data can be sorted or indexed by time, due to info about the time the telemetry applies to not being stored in the \gls{metadata} sent with the string: It only contains the times for when the transmission associated with the telemetry datum was initiated and completed, not the time the telemetry data itself applies to. These times may be significantly different, since the HYPSO satellite does not have continuous contact with a ground station.

\begin{code}
\inputminted[linenos=true,breaklines=true,bgcolor=codebg]{c}{./Code/eps_telemetry_struct_NAEPS001.txt}
\captionof{listing}{EPS telemetry \gls{struct} definition (C) from \cite{na_eps}}
\label{code:eps_struct}
\end{code}

Other inputs, such as image data and processed payload telemetry data, will likely also need to be integrated at a later time to allow accessing them through Open MCT. This is due to the flexibility it offers with regards to navigating and displaying data, and the convenience of eventually having all the data output from the satellite accessible in one place. This means that the implementation will need to be fairly flexible with regards to what inputs it accepts; raw telemetry from NanoAvionics will not be the only data input to the system in the long-term, which needs to be accounted and planned for from the beginning.

\subsection{Outputs}
Open MCT needs a certain minimum of output data from the implemented system to work, in addition to custom JavaScript plugins to map these outputs to inputs Open MCT can understand. The implemented system must provide past telemetry values plus present telemetry values to these plugins with a reasonably short delay after they are received, and also provide value \gls{metadata} so Open MCT can know how to display them.

\begin{code}
\inputminted[linenos=true,breaklines=true,bgcolor=codebg]{javascript}{./Code/telemetry_metadata_sample.json}
\captionof{listing}{Open MCT telemetry value metadata sample for \inlinecode{vBatt}}
\label{code:mct_metadata}
\end{code}

A sample of the practical minimum amount of \gls{metadata} Open MCT needs to display a visualisation for one telemetry value (here \inlinecode{vBatt}, the battery voltage, as stored in Source Code \ref{code:eps_struct}) is shown in Source Code \ref{code:mct_metadata}. This illustrates that there is a fair bit of work that needs to be done to go from input to output in the required implementation, at least for telemetry from the NanoAvionics server.

Another important note regarding the implementation's output is that other groups within HYPSO have also expressed a desire for access to unpacked telemetry data, which means that there should be a well-documented way to request or export its output data outside Open MCT.

\subsection{Requirements}
A concise overview of the interpreted requirements the implementation should fulfil is shown below.

\subsubsection{Functional Requirements}
\begin{enumerate}
  \item[1.0] Should display telemetry from NanoAvionics (NA) telemetry web server
  \item[1.1] Should display historical telemetry data from NA
  \item[1.2] Should display near-realtime telemetry data from NA
  \item[2.0] Should display telemetry from other sources
  \item[3.0] Should deliver or export processed telemetry outside Open MCT
  \item[4.0] Should be modular and expandable
  \item[5.0] Tests should have 75\% or better code coverage
\end{enumerate}

\subsubsection{Non-Functional Requirements}
\begin{enumerate}
  \item[6.0] Integrating new telemetry sources should require minimal extra work
  \begin{itemize}
    \item[Note:] \say{Minimal extra work} here implies that it should be well-documented how to add new telemetry sources, and that contact with new code and the need to understand implementation details of Open MCT should be minimised. The implementation should abstract away the connection to Open MCT from the user that is implementing the new telemetry source.
  \end{itemize}
  \item[7.0] Accessing telemetry data from outside Open MCT should require minimal extra work
  \begin{itemize}
    \item[Note:] This implies exports should be easy to acquire, and in a format that's easy to read and parse.
  \end{itemize}
  \item[8.0] Response times should be below one second for all \gls{frontend} actions
  \item[9.0] End-to-end delay should be less than one minute
  \begin{itemize}
    \item[Note:] This means that the time between new telemetry being available from NanoAvionics to it showing up in Open MCT should be less than one minute.
  \end{itemize}
  \item[10.0] Implementation should introduce a minimum of new programming languages, concepts and frameworks to the HYPSO project, to assist with code adoption and maintenance by existing developers
\end{enumerate}