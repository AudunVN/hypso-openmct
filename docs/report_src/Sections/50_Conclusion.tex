\begin{comment}

\end{comment}

\section{Conclusion}
The proposed and subsequently implemented system (named \Gls{depot}) meets the functional and non-functional requirements specified in Section \ref{scope} well, and should work in practice as a good basis for a mission planning and satellite troubleshooting tool. However, further work and user testing is needed to verify its performance as a practical telemetry visualisation system in real-world situations, and to make it ready for practical use.

Despite this, the system is already a significant improvement over the previous telemetry analysis data flow. This required manually using a command line tool to unpack and convert telemetry samples, before importing the raw data in an external tool such as Excel or Matlab for plotting and analysis if necessary. \Gls{depot} eliminates these manual steps, and allows real-time and historical telemetry values to be viewed directly in Open MCT, which makes mission monitoring and planning easier for ground station operators. In addition, the raw telemetry values can be exported quickly from Open MCT if further analysis or processing using external tools is required.

The process of implementing \Gls{depot} could have been streamlined by doing further research about the exact input and output requirements Open MCT has before finalising the design specification draft and class diagrams. Lack of detailed knowledge here was the factor that led to most of the unforeseen design changes and delays in implementing the first functional version of \Gls{depot}. Part of this is likely due to it being somewhat difficult to go from the Open MCT API specification - the main up-to-date source of documentation as of June 2020 - to a working Open MCT-based data visualisation system.