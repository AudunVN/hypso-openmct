\begin{comment}

\end{comment}

\section{Conclusion}
The proposed and subsequently implemented system (named MCT \Gls{depot}) seems to meet the functional and non-functional requirements specified in Section \ref{scope} fairly well, and should work in practice as a good basis for a mission planning and satellite troubleshooting tool. However, further work and user testing is needed to verify its performance as a practical telemetry visualisation system in real-world situations, and to make it ready for practical use.

The process of implementing it could have been streamlined by doing further research about the exact input and output requirements Open MCT has before finalising the design specification draft and class diagrams. Lack of detailed knowledge here was the factor that led to most of the unforeseen design changes and delays in implementing the first functional version of \Gls{depot}. Part of this is likely due to it being somewhat difficult to go from the Open MCT API specification - the main up-to-date source of documentation as of June 2020 - to a working Open MCT-based data visualisation system.

%\todo[inline]{Expand conclusion}

%\todo[inline]{Summarise key insights from the design, testing, development process. Write something about experiences.}

%\todo[inline]{The conclusion should discuss briefly if your specific goals were met (not only requirements). Are there any key scientific insights? Be specific wrt. avenues of improvement and refinement. Not only user testing and realistic use. Be specific about insight into NA db and Open MCT, strengths and weaknesses. Insights, improvements also outside your system.}